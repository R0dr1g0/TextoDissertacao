%%=====================================================================
%\chapter{Ferramenta para modelagem \aladim}
%\label{editor}
%%=====================================================================
%
%Como já  dito, os elementos  da linguagem \aladim\ foram  definidos de
%acordo  com  a  \se~\cite{deSouza:2005},  o  ciclo  de  design  usando
%\aladim\ segue aquele  proposto por \citeonline{Sharp:etal:2007} e que
%o desenvolvimento da ferramenta para modelagem empregou os padrões que
%compõem  a  abordagem  \mda~\cite{OMG:MDA:2001}.   Dessa  forma,  este
%capítulo apresenta o  {\em Editor \aladim} e descreve  como se deu seu
%desenvolvimento.
%
%%=====================================================================
%\section{Requisitos da ferramenta}
%\label{aladimEditor}
%%=====================================================================
%
%Como  visto na  seção~\ref{generalVision},  \aladim\ possui  elementos
%diagramáticos  e  textuais.    Contudo,  independente  se  textual  ou
%diagramático,  o suporte  ferramental para  auxiliar na  utilização de
%determinada linguagem, sempre foi visto como um importante diferencial
%para seus  usuários.  Seguindo esta perspectiva,  foi desenvolvida uma
%ferramenta de suporte à linguagem, o {\em Editor \aladim}.  Para isso,
%o  primeiro passo  foi especificar  um  conjunto de  requisitos que  a
%ferramenta deveria atender.  Tais requisitos são listados a seguir:
%
%\begin{enumerate}
% 
% \item  {\em Permitir a  edição de  diagramas}: A  ferramenta deveria
%    permitir ao designer a construção de modelos de maneira intuitiva,
%    muito semelhante aos editores  gráficos mais comuns, empregado uma
%    área de  edição e uma  paleta de elementos da  linguagem. Bastando
%    arrastar os componentes do diagrama da paleta para área de edição.
%
%  \item  {\em  Suportar   uma  notação  processável}:  Considerando  a
%    possibilidade de  um modelo  \aladim\ ser empregado  em processos,
%    como a  geração automática de  interface, por exemplo.   Até mesmo
%    para auxiliar no próximo requisito.
%
%  \item  {\em   Validar  sintaticamente  o   diagrama}:  As  vantagens
%    esperadas   pelo   suporte   de   um   editor   gráfico,   estarão
%    comprometidas, se  o designer tiver  um esforço muito  grande para
%    checar visual e mentalmente se seu diagrama está correto. Assim, é
%    imprescindível que a ferramenta garanta ao designer que o diagrama
%    que ele está construindo esteja sintaticamente correto.
%
%\end{enumerate}
%
%Buscando   atender  aos   requisitos   estabelecidos,  optou-se   pelo
%desenvolvimento  da   ferramenta  sob  a   forma  de  plugin   para  o
%Eclipse\footnote{http://www.eclipse.org},  que  é  uma  plataforma  de
%desenvolvimento  que  fornece uma  gama  de  serviços,  dentre eles  a
%criação de  plugins que podem ser carregados,  integrados e executados
%dentro do  próprio ambiente.   A plataforma, tecnologias  e atividades
%desenvolvidas  para  criação do  Editor  \aladim\  são apresentados  a
%seguir.
%
%%=====================================================================
%\section{Tecnologias utilizadas}
%\label{editorEclipse}
%%=====================================================================
%
%A Fundação Eclipse possui vários projetos, onde são criadas e mantidas
%várias  ferramentas  integradas. Dentre  os  vários  projetos, o  {\em
%  Eclipse Modeling Project}\footnote{http://www.eclipse.org/modeling/}
%tem  o foco  na evolução  e promoção  das técnicas  de desenvolvimento
%baseado em  modelos, dentro da  comunidade de usuários  da plataforma.
%Para isso, ele provê  um conjunto unificado de frameworks, ferramentas
%e padrões de implementações~\cite{Steinberg:etal:2008}.
%
%O  \emf~({\em \Emf})~\cite{Budinsky:etal:2003}  é  a base  tecnológica
%para o {\em projeto modeling}, ele oferece toda infraestrutura para se
%metamodelar,  instanciar  e   manipular  modelos.   A  linguagem  para
%definição de metamodelos no \emf\ é  chamada de {\em Ecore}, que é uma
%implementação  para um subconjunto  do \mof.   O \emf\  possibilita ao
%usuário  gerar, a partir  de algum  modelo Ecore,  código java  (sob a
%forma de plugins  para a própria plataforma) que  irão possibilitar ao
%usuário  a criação  e manipulação  de  modelos em  conformidade com  o
%metamodelo  estabelecido,  possibilitando  inclusive  a  validação  do
%mesmo.
%
%No  \emf\ não  há necessidade  de uma  metodologia específica  para se
%definir o  metamodelo, pois, ele gera automaticamente  modelos Ecore a
%partir de uma série de fontes como por exemplo, um diagrama \uml~({\em
%  \Uml}), arquivos \xml~({\em \Xml})  e código java anotado.  Contudo,
%o suporte para criação e  manipulação, oferecida é feito por no máximo
%um  editor de  árvore,  onde  os elementos  do  modelo são  incluídos,
%manipulados e removidos, sob a forma de nós filhos e irmão ao longo da
%árvore.
%
%Já o \gef~({\em \Gef})~\cite{Majewski:2004} é um framework que permite
%a criação de  editores gráficos na plataforma, através  de definição e
%customização  dos  elementos que  irá  compor  o  diagrama, tais  como
%figuras,  nós,  nós filhos,  arestas  e  compartimentos.  Os  editores
%criados também são  incorporados à plataforma sob a  forma de plugins.
%Contudo, o \gef\ oferece  suporte apenas à representação dos elementos
%gráficos, deixando todo o controle do modelo por trás do diagrama, sob
%o controle do usuário.  Durante  muito tempo isso foi feito através de
%\pojo~({\em \Pojo}),  ou seja,  toda essa implementação  precisava ser
%feita de maneira manual.
%
%É nesse contexto que  surge o \gmf~\cite{Plante:2006} para facilitar a
%integração desses dois frameworks (\emf\ e \gef).  Assim é possível ao
%usuário   desenvolver   um  completo   editor   gráfico,  através   da
%especificação  de  quatro  modelos:   de  domínio,  de  definição  dos
%elementos  gráficos,  de  definição  da  paleta de  ferramentas  e  de
%mapeamento    entre   os   outros    modelos,   como    ilustrado   na
%figura~\ref{fig:GMF:Workflow}.   O modelo  de  domínio corresponde  ao
%metamodelo  que  será  usado  para  instanciar  os  modelos  que  irão
%alimentar a visão ou diagrama  \gef, no caso da linguagem \aladim, seu
%metamodelo será apresentado na seção~\ref{aladimMetamodel}, a seguir.
%
%\figura{Fluxo     de    atividades     do     \gmf,    extraído     de
%  \citeonline{Plante:2006}.}
%       {fig:GMF:Workflow}
%       {gmf-workflow.eps}
%       {0.500}
%       {0}
%
%Dessa forma, para o desenvolvimento  do {\em Editor \aladim}, usando o
%{\em Eclipse Modeling}, o principal  recurso usado foi \gmf\ que, como
%visto,   é   responsável   por   prover  componentes   generativos   e
%infraestrutura de tempo de execução para o desenvolvimento de editores
%gráficos baseados  no \emf\  e usando o  \gef. Integrando todos  sob a
%forma de plugins que podem ser integrados à própria plataforma.
%
%%=====================================================================
%\section{Metamodelo da linguagem}
%\label{aladimMetamodel}
%%=====================================================================
%
%Entre  as seções  \ref{basicInteractions}  e \ref{transitions},  foram
%apresentados, informalmente  (não processável), a  sintaxe e semântica
%dos  elementos  da  linguagem   \aladim.   Agora,  nesta  seção,  será
%apresentada o  metamodelo (processável) da  linguagem, especificando a
%sintaxe  e semântica  de seus  elementos.  Considerando  os requisitos
%estabelecidos  para  a  ferramenta  e  as  tecnologias  utilizadas,  o
%metamodelo é  especificado usando um  modelo, que é a  linguagem usada
%para  representar  modelos  no  \emf,  isto é,  metamodelos.   Como  a
%linguagem de modelagem Ecore é  especificada no \emf, o Ecore se torna
%um metamodelo de si mesmo~\cite{Budinsky:etal:2003}.
%
%Um modelo  Ecore pode ser  criado de várias  formas no \emf,  como por
%exemplo, a importação de diagramas \uml, \xml\ e código java anotação.
%A forma mais básica de manipulação  é através de um editor de árvores.
%Adicionalmente  é  possível  manipular  um  modelo  Ecore  de  maneira
%diagramática, sendo  que essa  forma foi utilizada  para a  criação do
%metamodelo  da  linguagem  \aladim,  cuja  representação  é  feita  na
%figura~\ref{fig:aladim:metamodel}.
%
%\figura{Metamodelo da linguagem \aladim.}
%       {fig:aladim:metamodel}
%       {aladim-metamodel.eps}
%       {0.750}
%       {0}
%
%Um {\em modelo de interação} é composto por dois conjuntos, um de {\em
%  elementos}  e  outro das  {\em  transições}  entre esses  elementos.
%Dessa  forma,  são necessários  pelo  menos  dois  elementos para  uma
%transição.  Um elemento é uma abstração para {\em função da aplicação}
%e  {\em  espaços},  que  é   outra  abstração  para  {\em  espaços  de
%  interação}, {\em pontos iniciais} e  {\em finais} para a interação e
%{\em espaços de interação referenciados} e {\em aplicações externas}.
%
%Um espaço de interação é composto por, pelo menos, uma {\em interação}
%que é uma abstração para {\em interações básicas} e {\em operadores de
%  interação}.  Os  operadores podem conter novas interações,  o que os
%tornam recursivos. Um  operador é uma abstração para  {\em join}, {\em
%  sequence}, {\em choose}, {\em  repeat} e {\em combine}.  Assim, como
%uma interação básica  é uma abstração para {\em  select}, {\em enter},
%{\em perceive} e {\em activate}.
%
%Uma transição é  uma abstração para {\em ação},  {\em navegação}, {\em
%  reação de sucesso}, {\em reação  de falha} e {\em sincronização}. As
%transições, quando  disparadas pelo usuário, devem  identificar o nome
%da informação  associada a interação  básica que disparou.   Já quando
%disparadas  pelas  funções  da  aplicação,  podem  levar  aos  espaços
%informações  sobre  resultados   ou  falhas  no  processamento.   Toda
%transição possui uma origem e um destino.
%
%O \emf\ oferece  suporte para a geração automática  de código capaz de
%editar  e  validar  um  modelo  de acordo  com  as  especificações  do
%metamodelo. A validação consiste, dentre outras coisas, na verificação
%de todas  as associações  estabelecidas no metamodelo,  incluindo suas
%cardinalidades.    Contudo,  algumas  situações   em  que   apenas  as
%associações, especializações e  cardinalidades não são suficiente para
%especificar   determinadas  restrições,  é   possível  fazer   uso  de
%\ocl~({\em  \Ocl}), que é  uma linguagem  formal usada  para descrever
%expressões em modelos~\cite{OMG:OCL:2012}.
%
%Como visto  na descrição das  transições (seção~\ref{transitions}), os
%diferentes  tipos de  transições têm  algumas  restrições específicas.
%Além  disso, o  metamodelo  da figura~\ref{fig:aladim:metamodel},  não
%especifica isso.  Dessa  forma, considerando que de acordo  com o tipo
%de transição,  as origens e  destinos devem ser de  tipos específicos,
%por  exemplo,  em uma  {\em  ação}  a origem  deve  ser  um espaço  de
%interação e  o destino uma  função da aplicação, essas  restrições são
%especificas   através  de  invariantes   definidos  nas   classes  que
%especializam as abstrações, usando  \ocl, que também são traduzidos em
%código  para validação  durante  a modelagem  usando  o editor.   Este
%invariantes são descritos a seguir:
%
%\begin{itemize}
%
%  \item {\em  Action}: Especialização de transição que  liga um espaço
%    de interação a uma função da aplicação:
%
%    \begin{itemize}
%
%      \item {\em  not self.source.oclIsTypeOf(ApplicationFunction)}: A
%        origem de uma ação não pode ser uma função da aplicação;
%
%      \item   {\em   self.target.oclIsTypeOf(ApplicationFunction)}:  O
%        destino uma ação deve ser uma função da aplicação;
%
%      \item   {\em  self.source.oclIsTypeOf(InteractionSpace)\\implies
%        self.source.oclAsType(InteractionSpace).getNestedBasicInteractions()\\->exists(i:
%        BasicInteraction  | i.information  = self.information)}:  Se a
%        origem de  uma ação for  um espaço de interação,  deve existir
%        uma interação básica em  seu interior responsável por disparar
%        a transição;
%
%    \end{itemize}
%
%  \item  {\em  Reaction}: Especialização  de  transição  que liga  uma
%    função da aplicação a um espaço de interação:
%
%    \begin{itemize}
%
%      \item   {\em   self.source.oclIsTypeOf(ApplicationFunction)}:  A
%        origem de uma reação deve ser uma função da aplicação;
%
%      \item {\em  not self.target.oclIsTypeOf(ApplicationFunction)}: O
%        destino de uma reação não pode ser uma função da aplicação;
%
%    \end{itemize}
%
%  \item  {\em Navigation}:  Especialização de  transição que  liga uma
%    função da aplicação a um espaço de interação:
%
%    \begin{itemize}
%
%      \item {\em  not self.source.oclIsTypeOf(ApplicationFunction)}: A
%        origem de uma navegação não pode ser uma função da aplicação;
%
%      \item {\em  not self.target.oclIsTypeOf(ApplicationFunction)}: O
%        destino de uma navegação não pode ser uma função da aplicação;
%
%      \item   {\em  self.source.oclIsTypeOf(InteractionSpace)\\implies
%        self.source.oclAsType(InteractionSpace).getNestedBasicInteractions()\\->exists(i:
%        BasicInteraction  | i.information  = self.information)}:  Se a
%        origem  de uma  navegação  for um  espaço  de interação,  deve
%        existir uma  interação básica em seu  interior responsável por
%        disparar a transição;
%
%    \end{itemize}
%
%  \item {\em Synchronization}: Especialização de transição que liga um
%    espaço de interação a uma função da aplicação:
%
%    \begin{itemize}
%
%      \item {\em  not self.source.oclIsTypeOf(ApplicationFunction)}: A
%        origem  de  uma  sincronização  não  pode ser  uma  função  da
%        aplicação;
%
%      \item   {\em   self.target.oclIsTypeOf(ApplicationFunction)}:  O
%        destino de uma sincronização deve ser uma função da aplicação;
%
%    \end{itemize}
%
%\end{itemize}
%
%Os resultados do processo de validação de um modelo \aladim, podem ser
%observados   na   figura~\ref{fig:aladim:editor},   através   de   uma
%janela/painel    específico    para    apresentação   dos    problemas
%ocorridos. Além disso,  os elementos do diagrama são  decorados com os
%indicadores desses problemas. No  caso apresentado na figura, os erros
%decorrem justamente pela  violação de algumas restrições especificadas
%nos {\em invariantes} descritos acima.
%
%%=====================================================================
%\section{Editor \aladim\ em execução}
%\label{editorRunning}
%%=====================================================================
%
%Como resultado,  de tudo que  foi descrito nas seções  anteriores, foi
%produzido o {\em Editor \aladim} que pode ser visualizado, em execução
%sob a  forma de  um plugin já  completamente integrado ao  Eclipse, na
%figura~\ref{fig:aladim:editor},  onde  os  componentes  principais  do
%editor,  distribuídos  sob a  forma  janelas  ou  painéis do  ambiente
%Eclipse, são descritos a seguir.
%
%A  primeira,  mais  à esquerda,  é  a  janela  do {\em  explorador  de
%  pacotes}, onde  ficam os projetos que empacotam  os diagramas.  Cada
%projeto é armazenado no sistema de arquivo sob a forma de uma pasta ou
%diretório. Já  os diagramas  são armazenados sob  a forma  de arquivo,
%sendo dois para cada diagrama.   Um arquivo ({\tt .aladim}) armazena o
%\xml\ do  modelo, isto  é as instâncias  dos elementos  do metamodelo,
%implementados usando o  \emf.  Outro arquivo ({\tt .aladim\_diagram}),
%armazena o \xml\ do diagrama,  isto é, os elementos visuais associados
%aos elementos do modelo, implementados usando o \gef.
%
%A  segunda janela, posicionada  ao centro,  é a  {\em área  de edição}
%propriamente  dita,  é nela  que  o  designer  desenha seus  diagramas
%\aladim.  Para  cada elemento  visual presente no  diagrama, desenhado
%através do \gef, o ambiente,  através do \gmf, persiste um elemento no
%modelo, com  auxilio do \emf.   As operações relacionadas  à aparência
%dos  elementos visuais  (cor, fonte,  posição, etc.)   são persistidas
%apenas  no  diagrama   (arquivo  .aladim\_diagram),  enquanto  que  as
%operações   conceituais,  como  adição   e/ou  remoção,   mudanças  de
%propriedades e  referências dos  elementos, são persistidas  no modelo
%(arquivo .aladim), sendo que o sincronismo entre esses dois arquivos é
%completamente garantido pelo \gmf.
%
%Na  janela  mais  à  direita  fica  a {\em  paleta}  de  elementos  da
%linguagem. No  estilo {\em drag-and-drop}, o  designer deve selecionar
%cada  elemento e  adicioná-los  ao diagrama,  sendo  editar na  região
%central. Os elementos da paleta estão organizados em cinco categorias,
%que se comprimem e se  expandem para apresentação dos elementos. Essas
%categorias são:
%
%\begin{itemize}
%
%  \item  {\em Geral}: Permite  adicionar os  elementos mais  gerais da
%    linguagem, como  espaço de interação  ou função da  aplicação, por
%    exemplo.
%
%  \item {\em  Operador}: Permite adicionar, a um  espaço de interação,
%    os operadores como junção ou escolha, por exemplo.
%
%  \item {\em Interação}: Permite adicionar as interações básicas, como
%    entre com uma informação ou selecione uma informação, por exemplo.
%
%  \item {\em Transição}: Permite  adicionar as transições, com ação ou
%    navegação, entre os elementos gerais da linguagem.
%
%\end{itemize}
%
%Na parte inferior ficam várias janelas, entre elas a de edição de {\em
%  propriedades}  para elementos  selecionados no  modelo e  a  de {\em
%  problemas}  encontrados  no  modelo  que  decorrem  do  processo  de
%validação      do     mesmo.       No     caso      apresentado     na
%figura~\ref{fig:aladim:editor},  são  apresentados  quatro  problemas,
%propositalmente deixados no modelo.   Observe que além da listagem dos
%problemas com a indicação do  tipo de problema e sua localização, eles
%também são assinalados no diagrama através de decoração adicionados ao
%componente visual corresponde  ao elemento do modelo, que  ao passar o
%mouse sobre ele irá informar o tipo de problema ocorrido.
%
%\figura{Editor \aladim\ em execução integrado ao Eclipse.}
%       {fig:aladim:editor}
%       {aladim-editor.eps}
%       {0.555}
%       {-90}
%
%%=====================================================================
%\section{Considerações sobre o capítulo}
%\label{editorSynthesis}
%%=====================================================================
%
%Este  capítulo  apresentou  o  {\em Editor  \aladim},  uma  ferramenta
%desenvolvida  para permitir  a construção  de  diagramas representando
%modelos  nessa  linguagem. Seu  objetivo  é  auxiliar  no processo  de
%comunicação   do  designer  com   os  demais   membro  da   equipe  de
%desenvolvimento, especialmente os desenvolvedores.
%
%Desenvolvido  usando a  abordagem  \mda\ e  concretizado como  plugins
%integrados ao Eclipse, a  ferramenta permite usar todas as facilidades
%oferecidas pela plataforma.  Além  disso, dentro da abordagem \mda, os
%modelos \aladim\  estão num  nível \pim\ de  abstração, o  que permite
%que, usando  o metamodelo de \aladim, possam  ser produzidos processos
%automáticos  para  os  demais  níveis.  Possibilitando,  inclusive,  a
%geração automática de código da interface para diversas plataformas.
%
%Tudo   isso  demonstra   que  a   ferramenta  atende   aos  requisitos
%(seção~\ref{aladimEditor}) para os quais ela foi desenvolvida. Além da
%edição,  no  estilo  {\em  drag-and-drop}, dos  diagramas  \aladim,  a
%ferramenta  também possui  um editor  independente, que  possibilita a
%edição do modelo como uma  árvore. Sendo que a ferramenta mantém esses
%dois   arquivos  sincronizados   e  quaisquer   alterações  em   um  é
%automaticamente refletida no outro.
%
%%%% Local Variables: 
%%%% mode: latex
%%%% TeX-master: "main"
%%%% End: 
