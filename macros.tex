%=====================================================================
% Macros usadas no documento
%---------------------------------------------------------------------
% Para expressões em inglês:
% \en{hardware}, \en{software}
%---------------------------------------------------------------------
\newcommand{\en}[1]{
  \selectlanguage{english}
  \textit{#1}
  \selectlanguage{brazil}
}
%---------------------------------------------------------------------
% Definindo o comando figura
% 1 - Legenda da figura
% 2 - Rótulo para referência da figura
% 3 - Nome de arquivo que contém a figura
% 4 - Escala da figura
% 5 - Ângulo de rotação da figura
%---------------------------------------------------------------------
\newcommand{\figura}[5]{
  \begin{figure}[!htb]
    \begin{center}
      \includegraphics[scale=#4,angle=#5]{images/#3}
      \caption{#1}
      \label{#2}
    \end{center}
  \end{figure}
}
%---------------------------------------------------------------------
% Versão estendida para expandir em documentos com duas colunas
%---------------------------------------------------------------------
\newcommand{\figuraL}[5]{
  \begin{figure*}[!htb]
    \begin{center}
      \includegraphics[scale=#4,angle=#5]{images/#3}
      \caption{#1}
      \label{#2}
    \end{center}
  \end{figure*}
}
%---------------------------------------------------------------------
% Símbolos para 
%---------------------------------------------------------------------
\newcommand{\X}{~~$\bullet$}
\newcommand{\cf}{\ding{52}}
\newcommand{\vi}{\ding{56}}
\newcommand{\na}{{\bf --}}
%---------------------------------------------------------------------
% Siglas e significados
%---------------------------------------------------------------------
% OMG Stardarts
%---------------------------------------------------------------------
\newcommand{\omg}{OMG}
\newcommand{\Omg}{Object Management Group}
\newcommand{\mda}{MDA}
\newcommand{\Mda}{Model Driven Architecture}
\newcommand{\mof}{MOF}
\newcommand{\Mof}{Meta Object Facility}
\newcommand{\uml}{UML}
\newcommand{\Uml}{Unified Modeling Language}
\newcommand{\ocl}{OCL}
\newcommand{\Ocl}{Object Constraint Language}
\newcommand{\xmi}{XMI}
\newcommand{\Xmi}{\xmi\ Metadata Interchange}
\newcommand{\xml}{XML}
\newcommand{\Xml}{Extensible Markup Language}
\newcommand{\qvt}{QVT}
\newcommand{\Qvt}{Query/View/Transformation}
%---------------------------------------------------------------------
% MDA models
%---------------------------------------------------------------------
\newcommand{\cim}{CIM}
\newcommand{\Cim}{Computation Independent Model}
\newcommand{\pim}{PIM}
\newcommand{\Pim}{Platform Independent Model}
\newcommand{\psm}{PSM}
\newcommand{\Psm}{Platform Specific Model}
%---------------------------------------------------------------------
% Eclipse Modeling
%---------------------------------------------------------------------
\newcommand{\emf}{EMF}
\newcommand{\Emf}{Eclipse Modeling Framework}
\newcommand{\gef}{GEF}
\newcommand{\Gef}{Graphical Editing Framework}
\newcommand{\gmf}{GMF}
\newcommand{\Gmf}{Graphical Modeling Framework}
\newcommand{\pojo}{POJO}
\newcommand{\Pojo}{Plain Old Java Object}
%---------------------------------------------------------------------
% Model-based
%---------------------------------------------------------------------
\newcommand{\diubm}{DIUBM} 
\newcommand{\Diubm}{Desenvolvimento de Interface de Usuário Baseado em
  Modelos}
\newcommand{\mbuid}{MBUID}
\newcommand{\Mbuid}{Model-Based User Interface Development}
\newcommand{\mdd}{MDD}
\newcommand{\Mdd}{Model Driven Development}
%---------------------------------------------------------------------
% Gerais de IHC
%---------------------------------------------------------------------
\newcommand{\aladim}{ALaDIM}
\newcommand{\Aladim}{Abstract Language  for Description of Interactive
  Message}
\newcommand{\molic}{MoLIC}
\newcommand{\Molic}{Modeling Language for Interaction as Conversation}
\newcommand{\imml}{IMML}
\newcommand{\Imml}{Interactive Message Modeling Language}
\newcommand{\ihc}{IHC}
\newcommand{\Ihc}{Interação Humano-Computador}
\newcommand{\chimy}{CHI}
\newcommand{\Chimy}{Computer-Human Interaction}
\newcommand{\ec}{Engenharia Cognitiva}
\newcommand{\ta}{Teoria da Ação}
%---------------------------------------------------------------------
% Software Engineering
%---------------------------------------------------------------------
\newcommand{\rup}{RUP}
\newcommand{\Rup}{Rational Unified Process}
\newcommand{\es}{ES}
\newcommand{\Es}{Engenharia de Software}
\newcommand{\ese}{ESE}
\newcommand{\Ese}{Engenharia de Software Experimental}
\newcommand{\gui}{GUI}
\newcommand{\Gui}{Graphical User Interface}
%---------------------------------------------------------------------
% Semiotic Engineering
%---------------------------------------------------------------------
\newcommand{\se}{Engenharia Semiótica}
\newcommand{\mac}{MAC}
\newcommand{\Mac}{Método de Avaliação da Comunicabilidade}
\newcommand{\mis}{MIS}
\newcommand{\Mis}{Método de Inspeção Semiótica}
%---------------------------------------------------------------------
% Gerais
%---------------------------------------------------------------------
\newcommand{\cdn}{CDN}
\newcommand{\Cdn}{Cognitive Dimensions of Notations}
\newcommand{\gqm}{GQM}
\newcommand{\Gqm}{Goal/Question/Metric}


\newcommand{\linkert}{\\(~~) Pouquíssimo (~~) Pouco (~~) Neutro (~~) Muito
  (~~) Demasiado}
%=====================================================================
