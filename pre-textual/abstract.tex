%=====================================================================
% Resumo e Abstract
%=====================================================================
\begin{abstract}
\selectlanguage{english}

\vfill

{
  \fontsize{12}{14}
  \selectfont

%  \noindent   In  the  Semiotic   Engineering  theory,   the  designer
%  communicates to users, throughout the system's interface, {\em what}
%  and {\em how}  they can do to use  the system's functionalities.  In
%  that  perspective, the  interface  is an  {\em interactive  message}
%  because,  that communicates  these  aspects to  the  user, and  also
%  allows the  communication between user and  system.  The development
%  of  interactive  systems  involves  several  professionals  and  the
%  communication between  them normally uses common  artifacts, such as
%  models,  that drive  the development  process.  In  the model-driven
%  development  approach, the  interaction  model is  an artifact  that
%  includes the  most of the aspects  related to what and  how the user
%  can do  while he/she interacting with the  system.  Furthermore, the
%  interactive  model may  be used  to identify  usability  problems at
%  design time.   Therefore, the central problematic  addressed by this
%  thesis is twofold.  In the first place, the interaction modeling, in
%  a communicative perspective, in  which the developer is considered a
%  translator  that  codes  the   interactive  message  into  the  user
%  interface.  In  the second place, the  anticipated identification of
%  usability problems, that aims to reduce the application final costs.
%  To   achieve    these   goals,   this   work    presents   (i)   the
%  \aladim\  language, which  is grounded  on Semiotic  Engineering and
%  aims  to help  the designer  on the  conception,  representation and
%  validation   of   his   interactive   message   models;   (ii)   the
%  \aladim\ editor, which was built using the \emf~({\em \Emf}) and its
%  standardized  technologies  by  \omg~({\em  \Omg});  and  (iii)  the
%  \aladim\   inspection   method,   which   allows   the   anticipated
%  identification   of  usability   problems  using   \aladim\  models.
%  \aladim\  language  and   editor  were  respectively  specified  and
%  implemented  using the  \omg\  standards  and they  can  be used  in
%  \mda~({\em  \Mda})  activities.   Beyond  that,  we  evaluated  both
%  \aladim\  language and  editor using  a \cdn~({\em  \Cdn}) analysis.
%  Finally,  this  work  reports   an  experiment  that  validated  the
%  \aladim\ inspection method.

}
\vfill

{
  \fontsize{12}{15}
  \selectfont

  \begin{espacosimples}
    \noindent {\bf Advisor}: \orientadorA\\
    \noindent {\bf Area of concentration}: \areaEMingles\\
    \noindent {\bf Keywords}: \palavrasEMingles\\
    \noindent {\bf Number of pages}: \pageref{frontpages} + \pageref{final}
  \end{espacosimples}

}

\selectlanguage{brazil}
\end{abstract}
