%=====================================================================
% Resumo
%=====================================================================
%% O  resumo deve  ter frases  indicando:  o contexto;  o problema  ou
%% questão  de pesquisa;  a  justificativa para  a  sua relevância;  a
%% solução que vai ser dada; quais as contribuições e como ela vai ser
%% verificada e validada (revisão do Jair).
%---------------------------------------------------------------------
\begin{resumo}

\vfill

{
  \fontsize{12}{14}
  \selectfont

%  \noindent A teoria da Engenharia Semiótica, considera que o designer
%  comunica ao  usuário, por  meio da interface,  {\em o que}  ele pode
%  fazer e  {\em como} deve  agir para usufruir das  funcionalidades do
%  sistema.  Dessa  forma, a  interface é vista  com uma  {\em mensagem
%    interativa} porque,  além de comunicar estes  aspectos ao usuário,
%  também possibilita  a comunicação entre  o usuário e o  sistema.  No
%  desenvolvimento de um  sistema interativo, estão envolvidos diversos
%  profissionais  e a  comunicação  entre eles  se dá,  principalmente,
%  através do  compartilhamento de artefatos comuns,  como modelos, por
%  exemplo.    O  uso   de   modelos  possibilita   uma  abordagem   de
%  desenvolvimento  na  qual os  modelos  são  usados  por designers  e
%  desenvolvedores.   Na  abordagem baseada  em  modelos,  o modelo  de
%  interação  é   um  artefato  que   cobre  a  maioria   dos  aspectos
%  relacionados  ao  ``o que''  e  ``como''  o  usuário pode  fazer  ao
%  interagir com  a aplicação.  Além disso,  ele pode ser  usado para a
%  identificação  de problemas de  usabilidade durante  o design  e não
%  apenas durante  os testes  de interface e  aceitação, o que  reduz o
%  impacto  nos  custos  de  processo.  Nesse  sentido,  este  trabalho
%  endereça duas questões.  A primeira  é a modelagem da interação, sob
%  uma perspectiva comunicativa, na qual o designer terá que contar com
%  o  desenvolvedor que  irá atuar  como  um tradutor  ao codificar  na
%  interface  sua  mensagem  interativa.   A  segunda  é  identificação
%  antecipada  de  problemas   de  usabilidade,  que  visa  contribuir,
%  principalmente, com  a redução dos custos  de desenvolvimento.  Para
%  isso,  este  trabalho  apresenta  (i)  a linguagem  \aladim,  que  é
%  fundamentada  na \se\  e  busca auxiliar  o  designer na  concepção,
%  representação  e   validação  dos  modelos  de   sua  {\em  mensagem
%    interativa}; (ii)  o editor \aladim,  que foi construído  usando o
%  \emf~({\em  \Emf}) e suas  tecnologias padronizadas  pelo \omg~({\em
%    \Omg});  e (iii)  o  método  de inspeção  \aladim,  que permite  a
%  identificação  antecipada  de problemas  de  usabilidade em  modelos
%  \aladim.  A  linguagem e  o editor \aladim\  foram, respectivamente,
%  especificada  e implementado, usando  padrões do  \omg\ e  podem ser
%  empregados  em  atividades  \mda~({\em  \Mda}).  Além  disso,  foram
%  avaliados  a linguagem  e o  editor \aladim,  através da  análise da
%  dimensões   cognitivas  de  ambos,   usando  o   \cdn~({\em  \Cdn}).
%  Finalmente, este  trabalho relata  um experimento para  validação do
%  método de inspeção.

}

\vfill

{
  \fontsize{12}{15}
  \selectfont

  \begin{espacosimples}
    \noindent {\bf Orientador}: \orientadorA\\
    \noindent {\bf Área de concentração}: \area\\
    \noindent {\bf Palavras-chave}: \palavras\\
    \noindent   {\bf  Número   de  páginas}:   \pageref{frontpages}  +
    \pageref{final}
  \end{espacosimples}

}

\end{resumo}
